\documentclass[twocolumn]{article}
\usepackage[utf8]{inputenc}
\usepackage{xcolor}
\usepackage{hyperref}
\hypersetup{
    colorlinks=true,
    linkcolor=olive,
    citecolor=violet,
    urlcolor=purple
}
\usepackage[
backend=biber,
style=apa
]{biblatex}
\addbibresource{bibliography.bib}

\title{Visualizing Color Space}
\author{Kyle Christopher McDermott}
\date{\today}

\begin{document}

\begin{titlepage}
\maketitle
\thispagestyle{empty} % Removes page number
\end{titlepage}

\twocolumn[
\section{Introduction} \label{sec:introduction}
Here are described methods for generating and, in particular, coloring diagrams of color space (chromaticity) and the visible spectrum.  Section \ref{sec:derivation} covers the derivation of the CIE $(x,y)$ chromaticity diagram starting from experimental data collected from human observers, through the derivation of cone (photoreceptor) sensitivities and the color matching functions used to determine chromaticity from light spectra.  Section \ref{sec:color_mixing} uses the chromaticity diagram to illustrate how a few primary colors can be mixed to create a range of colors between, including the sRGB standard which will be used here to determine chromaticity from display color and vice versa.  Section \ref{sec:chromaticity_coloration} presents multiple methods for the coloration of diagrams of chromaticity and of the visible spectrum and the key differences and considerations for selecting among them.  Section \ref{sec:color_temperature} discusses correlated color temperature - relating to black-body radiation and the Planckian locus which is often depicted on chromaticity diagrams.  Finally, Section \ref{sec:color_blindndess} illustrates the theory behind both generating test stimuli for, and filtering images to mimic the effects of, color blindness.\\ \\
By far the resource used most in the preparation of this document is the excellent database maintained by Andrew Stockman and his colleagues in the Color \& Vision Research Laboratory at the Institute of Ophthalmology, University College London (\url{http://www.cvrl.org/} or \url{http://cvrl.ucl.ac.uk/}).  The theory, key variables, and tabulated data used both as source and a check on my own computations are there thoughtfully laid out, and links to key pages on their website appear throughout Section \ref{sec:derivation}.
\section{Derivation of CIE $(x,y)$ Chromaticity} \label{sec:derivation}
]
Typically the color matching functions of CIE 1931 standard (\cite{cie1932commission}) are used to determine the $(x,y)$ chromaticity coordinate from a spectrum of light - doing so for a range of hypothetically purely monochromatic lights results in the “spectrum locus” curve depicted in chromaticity diagrams - and these are based on experiments which presented a relatively small 2$^\circ$ visual angle stimulus to experimental observers (\cite{wright1929re,guild1931colorimetric}).  However, the variables and tabulated data necessary to derive these color matching functions from experimental observations - and to verify my own computations at each step - are more readily available for a larger 10$^\circ$ stimulus (\cite{stiles1959npl}) (which originally contributed to the CIE 1964 standard).  The method detailed here ultimately results in color matching functions for the CIE 170-2 10$^\circ$ Observer standard.  The more typical CIE 1931 2$^\circ$ Observer standard color matching functions will be used in later sections primarily because the sRGB primaries are defined in that space.  As will be illustrated, the 2$^\circ$ and 10$^\circ$ chromaticity diagrams are similar enough to make the distinction important only for narrowly defined circumstances (i.e. experiments probing human vision).\\

\subsection{Color Matching Experiment} \label{sec:experiment}
Stiles \& Burch presented in 1959 data from 49 individual observers (including 4 experimenters who are represented twice for a total of 53 observations across two similar conditions) using a 10$^\circ$ Stimulus (\cite{stiles1959npl}).  They also presented data from 10 observers for a 2$^\circ$ stimulus, but this has been deemed too few from which to build a “standard observer”.  The Color \& Vision Research Laboratory have made the data available here (file download \href{http://www.cvrl.org/database/data/sb_individual/SB10_corrected_indiv_CMFs.xls}{link}); these are “corrected data” (see \href{http://www.cvrl.org/database/text/sb_individual/sb10_individual.htm}{here} (\cite{stiles1958average})).
\subsubsection{Stimuli \& Procedure}
The experiment presented observers with a circular stimulus 10$^\circ$ in diameter, bisected vertically.  On one side were presented a range of narrow-band lights from 14,000 $cm^{-1}$ to 25,500 $cm^{-1}$ (roughly 714 $nm$ to 392 $nm$, note the reversed order as the units are inverses of each other).  On the other side was a mixture of three “primary” lights at 15,500 $cm^{-1}$ (roughly 645 $nm$, and reddish), 19,000 $cm^{-1}$ (roughly 526 $nm$, and greenish), and 22,500 $cm^{-1}$ (roughly 444 $nm$, and bluish); the experimental data are “transformed” from the actual, physical values of 15,418 $cm^{-1}$, 18,997 $cm^{-1}$, and 22,456 $cm^{-1}$, respectively.  Observers’ task was to adjust the intensity of the three primary lights on one side to match the single light on the other side.\\ \\
As will be illustrated later, it is not always possible to mix three primaries in this way to create a match if the test stimulus is outside the range of colors achievable by a mixing of those primaries.  Therefore the experimenters allowed the primaries to be adjusted to “negative” intensities - this meant that instead of adding that light to the mixture of primaries, it was instead added to the test light.  Therefore, when one of the primaries was adjusted to a negative intensity, observers were modifying both sides of the stimulus to achieve a match.
\subsubsection{Data Visualization}

\section{Color Mixing} \label{sec:color_mixing}

\section{Chromaticity Diagram Coloration} \label{sec:chromaticity_coloration}

\section{Color Temperature} \label{sec:color_temperature}

\section{Color-Blindness} \label{sec:color_blindndess}
\twocolumn[
\printbibliography
]

\end{document}
